%\documentclass[14pt]{extarticle}
\documentclass[11pt]{article}
\usepackage{natbib}
\usepackage[dvipdfm,colorlinks=true,urlcolor=DarkBlue,linkcolor=DarkBlue,bookmarks=false,citecolor=DarkBlue]{hyperref}

\usepackage[pdftex]{graphicx}
\usepackage{fancyhdr}
\usepackage[T1]{fontenc}
\usepackage{palatino}
\usepackage[utf8]{inputenc}
%\usepackage[super]{nth}
\usepackage{setspace}
\usepackage{placeins}
\usepackage{subfigure}
\usepackage{multirow}
\usepackage{rotating}
\usepackage{marvosym}  % Used for euro symbols with \EUR
\newcommand{\HRule}{\rule{\linewidth}{0.5mm}}
\usepackage{longtable} %% Allows the use of the longtable format produced by xl2latex.rb
\usepackage{lscape} %% Allows landscape orientation of tables
\usepackage{appendix} %% Allows customization of the appendix properties
\setcounter{tocdepth}{2} %% Restricts the table of contents to the section header level entries only

\usepackage{geometry}
\geometry{letterpaper}
\usepackage{amsmath}
\usepackage[stable]{footmisc}

%% The following settings are for the listings environment in R
\usepackage{listings}
\usepackage[svgnames]{xcolor}
\usepackage{soul}
\sethlcolor{LightGoldenrodYellow}
\lstset{backgroundcolor=\color{LightYellow}}
\lstset{framextopmargin=6pt, framexbottommargin=6pt, framerule=4pt, rulecolor=\color{White}}

\title{\texttt{leghist}:\\ Automated legislative history analysis for
  R\thanks{Original version: 30 January 2012. Thanks to Adrienne Hosek for discussions on the
  comparative legislative processes for the United States Congress and
the European Parliament.}}
\author{Mark Huberty\thanks{Travers Department of Political Science,
    University of California, Berkeley. Contact:
    \url{markhuberty@berkeley.edu}.}~ and Hillary Sanders\thanks{University of
California, Berkeley.}}
\date{\today}

\graphicspath{{../figures/}}

\begin{document}
%\input{/Users/markhuberty/Documents/Technology/Documentation/Typesetting/prospectusTitle.tex}
\maketitle
%\doublespacing

\begin{abstract}
  We describe an integrated workflow for automated analysis of
  legislative history in the R language. Legislative history analysis
  often requires laborious hand-matching of proposed amendments and
  their authors to their destination in a final bill. \texttt{leghist}
  deploys natural language and textual analysis methods to automate
  matching, estimate match accuracy, model the
  semantic content of amendments, tabulate the rate of acceptance and
  rejection of amendments by legislative actor and topic, and visualize the
  resulting flow of information. We provide a theoretical
  overview of the workflow and suggest this model could be extended to
  other analogous forms of legal and political document analysis. We
  finally provide a vignette based on the European Union's Third
  Renewable Energy Directive, adopted in 2008. 
\end{abstract}

\tableofcontents 
\section{Introduction}
\label{sec:introduction}

Analysis of legislative history plays a central role in political
science. The data often present barriers to making straightforward, or
quantifiable progress. Few legislatures provide marked-up versions of final bills
indicating when and how proposed legislation was amended, or by
whom. Particularly for very large bills with many amendments,
identifying these important sources of information on legislative
influence and process constitute a very large search problem. 

The \texttt{leghist} package for R is designed to help analysts
address these barriers. \texttt{Leghist} provides computational tools for
processing text from proposed legislation and amendments, matching
sections of the final bill to their most likely origin in either the
proposed bill or subsequent amendments, and quantifying the degree and
form of variation among both successful and unsuccessful
amendments. We show, through hand-coded trials, that the
methods used here generate valid matches at a high degree of accuracy
in terms of both human-identified textual similarity and evidence from
the formal legislative record. Finally, we emphasize that the tools
provided here should work for other document history processes
analogous to the legislative history model we present below.

\section{How a bill becomes a law: a model}
\label{sec:how-bill-becomes}

This section presents an abstracted view of how a legislative proposal
evolves into a final bill through three three,
increasingly complex, processes. The biggest change imposed by the added
complexity concerns whether we should use supervised or unsupervised
methods to identify the origin of the components of the final
bill. While the most basic case permits a purely unsupervised
approach, the more complex cases benefit from some additional user
input. The tools provided in the
\texttt{leghist} package assist the analyst in making progress on both
the most basic case and each extension. 

Analysts interested in legislative history may raise a set of
questions about the history of piece of legislation:
\begin{itemize}
\item Which sections of the original bill survived to the final version?
\item Which amendments were adopted, and which weren't?
\item Who originated the adopted amendments, versus those not adopted?
\item What was the content of the amendments, and does that content
  categorize in systematic ways (i.e., did amendments focus on
  particular topics within the policy domain)?
\item Were there aspects of the final bill that have no good match,
  suggesting changes made behind the scenes (i.e., in the
  reconciliation process between the House and Senate in the United States Congress)
\end{itemize}


\subsection{The simple model: a pure legislative process}
\label{sec:simple-model:-pure}

Formally, we treat a bill as a document $D$ composed of $I$ sections
$d_i$. The legislative process begins with an initial bill $D_1$, and
concludes with a successful bill $D_2$. During the legislative
process, $J$ amendments $a_j \in A_J$ are proposed by some set of
legislative actors $L$, of which $A_K, 0 \leq K \leq J$ are
adopted. For $D_1, D_2,$ and $A_J$, we assume that the sections
$d_{1,i}, d_{2,i}$, and $a_j$ are all at the same level of
dis-aggregation (e.g. sentence, paragraph, article). The tools
presented here are agnostic to that level of dis-aggregation, but for
the rest of this paper we assume that sections are analogous to
paragraphs. This is consistent with the structure of amendments for
the European Parliamentary bills considered in section
\ref{sec:vign-2009-europ}.

For the simple case, then, the final bill is a
combination of the proposed bill and the proposed
amendments. We can represent that process as a functional form of $D_2
= f(D_1, A_J)$. Each section $d_{2,i}$ contains one and only one best
match in $D_1, A_j$. Given that, the challenge is to identify the sections
of the original bill and the amendments that comprise the final
bill. 

To do so, we can thus define a similarity measure $S$ to permit comparison
between each section $d_{2,i}$ of the final bill and its potential origins. For simplicity, we
assume that $S \in [0,1]$. For each final section $d_{2,i}$ we
construct a similarity vector $s_i$ between all possible matches in
$D_1, A_J$. The best match is thus $d \in arg \max_{s_i} S(D_1,
A_J)$. Under the assumption that $S$ is well-ordered, we can use
nearest-neighbor matching, since for each $d_{2,i}$ there is one and
only one best match.\footnote{A distance metric can also be
used, in which case the choice problem becomes $d \in arg \min_{s_i}
S(D_1, A_J)$.}

To construct the distance metric, we represent each section of $D_2$,
$D_1$, and $A_j$, as a term-frequency vector, where each element in
the vector represents the frequency with which one of the set of
unique words in the entire corpus occurs in this particular
section. Given the set of all bill sections in a term-frequency
matrix, we can measure the similarity between any two bill sections
using any continuously-valued similarity or distance metric.

\subsection{Extensions: fuzzy matches}
\label{sec:extens-fuzzy-match}

The first extension to this stylized legislative process involves a
set of fuzzy potential matches. Two possible developments will lead to
fuzzy matches. First, specific language in a bill, such as
cross-references to other bill sections, dates, or formal legal
language, may change without affecting the substantive semantic
content of the bill section and its best match. Second, the drafting
process may split apart and recombine sections prior to the final
bill, without necessarily affecting the origin of the legislative
language.\footnote{We also note that this set of changes may be
  introduced by the process of producing machine-readable text. Given
  that much of this text must be scraped by the analyst, this form of
  variation may be introduced by the scraping algorithm
  itself. Validating the algorithm for each bill represents a
  significant effort that we would like the analyst to avoid.} We would wish to identify both cases.

Nearest-neighbor matching with replacement handles both
problems. Matching with replacement permits a single amendment to
match to multiple sections of the final bill. This same approach also works for subtle changes in language
introduced during final drafting of the bill. For instance, consider
the section of a final bill (left) and its match in the initial bill
(right) presented as below:
\begin{quote}
  \begin{minipage}[t]{0.45\linewidth}
    to develop renewable energies to meet the commitment of the Com-
    munity to using 20\% renewable energies by 2020, as well as to
    develop other technologies contributing to the transition to a
    safe and sustainable low-carbon economy and to help meet the commitment of the Community to increase energy efficiency by 20\% by
    2020;
  \end{minipage}
  \begin{minipage}[h]{0.08\linewidth}
  \end{minipage}
  \begin{minipage}[t]{0.45\linewidth}
    to develop renewable energies to meet the commitment of the Com-
    munity to using 20\% renewable energies by 2020, and to meet the
    commitment of the Community to increase energy efficiency by 20\%
    by 2020;
  \end{minipage}
\end{quote}

Here, the use of nearest-neighbor matching does not require the exact
match between the final bill section and its candidate
sources. Rather, so long as the similarity measure produces a
well-ordered similarity vector, the match will return the correct value.

It is important to note one situation in which this matching process
breaks down. Introduction of the same amendment by multiple
legislative actors, or at multiple stages of the legislative process, will produce more than one match, since the
similarity vector is no longer well-ordered. Absent additional
information, it is impossible to determine which is the ``correct''
match from this set. Thus the algorithm may return the correct
\textit{semantic} origin without necessarily returning the correct
\textit{procedural} origin. Presently, \texttt{leghist} breaks such
ties by choosing the last version of the amendment submitted, on the
supposition that an amendment would not be re-submitted if it had been
accepted at an earlier stage of the legislative process. We will return to this issue when we discuss
supervised processes.

\subsection{Extensions: negotiated content and revisions}
\label{sec:extens-negot-cont}

Finally, some legislative processes may not require that all changes
to a bill be introduced as formal amendments. Examples here include the
reconciliation process in the United States Congress, and the
co-decision negotiation process between the European Parliament and Council. In
this case, substantive changes to a bill may occur without any
matching record in either the original bill or the proposed
amendments. 

Given that the nearest-neighbor process defined above will return some
match even if no actual match exists, handling this case requires a
filter for ``poor'' matches. We implement this process with a filter
on the similarity measure itself: for matches whose similarity
measure $s$ is below a user-supplied threshold $T$, we reject the
match and instead match the section to itself. This ``false'' match
can aid the identification of elements in a bill that
came out of negotiated rather than formal amendment processes. 

Setting the threshold poses the precision/recall trade-off. A low
threshold level will match 100\% of the actual matches (high
recall), while returning many false positives (low precision). A high
threshold level will do the reverse. \texttt{Leghist} treats this
trade-off as a supervised learning problem in which the optimal $T$ is
derived from a set of user-coded data. We provide tools to automate
the hand-matching of a random subset of the final bill to its
potential sources. Based on that hand coding, the proper threshold
value can be chosen to maximize the accuracy of the matching
algorithm. We provide the capability to estimate the optimal $T$ based
on both overall accuracy and the precision/recall trade-off.

\subsection{Extensions: influence estimates}
\label{sec:extens-infl-estim}

\cite{tsebelis2001legislative} suggest two measures of actor influence
in the legislative process. We can describe these as:

\begin{enumerate}
\item \textit{Semantic impact} measures whether the actors are
  successful at introducing substantively new content to bills (as
  opposed to simply tweaking existing content). An example here would
  be expanding an agriculture subsidy bill to cover humane treatment
  of animals.
\item \textit{Semantic completeness} measures whether proposed
  amendments were adopted in whole or only in part, regardless of
  whether they introduced new substance or modified existing content.
\end{enumerate}

\textit{Semantic impact} can be treated as a cluster analysis
problem. We treat the entire corpus of bills and amendments (a corpus
of length $D_1 + D_2 + A_J$) as representing $K$ latent topics. Each
section $d_i, a_j$ can be assigned to one of these $K$
topics. Substantively new content is thus a set of topics $K^* \in
D_2, K^* \not \in D_1$. Legislative actors whose contributions
constitute a higher proportion of $K^*$ are identified as having
greater substantive impact on outcomes. 

\textit{Semantic completeness} can be derived directly from the
distribution of the similarity or distance measures for the accepted
amendments from each legislative actor. Higher similarity
metric values indicate amendments adopted almost verbatim; lower
values suggest partially-adopted amendments. 

Finally, we may also wish to identify \textit{contested} policy
domains. To the extent that contestation appears in the written
legislative record, we could abstract it as follows: A section
of the final bill $d_{2,i}$, is considered more contested if:

\begin{enumerate}
\item The distribution of its similarity metric $S$ is bimodal, with a
  small cluster of very close matches and a large body of poor matches
\item The small cluster of close matches comes from a more diverse set
  of legislative actors
\end{enumerate}

Condition (1) suggests how we can programatically identify
contestation from the distribution of the similarity metric. Condition
(2) then eliminates those cases where one legislative actor
continually submitted the same amendment despite having little actual
success. Together, these suggest a mechanism for inferring
contestation from data.

\section{The technical implementation}
\label{sec:legisl-sect-match}

In \texttt{leghist}, we implement this model in five stages:\footnote{
Code implementing \texttt{leghist} is available at \url{https://github.com/markhuberty/leghist}.}
\begin{enumerate}
\item Nearest-neighbor matching of the final bill to candidate origins
\item Semi-supervised learning of the right threshold values
\item Synthetic bill construction
\item Semantic clustering and impact analysis
\item Visualization and presentation
\end{enumerate}


\subsection{Distance matching and estimation}
\label{sec:dist-match-estim}

Given the model presented in section \ref{sec:how-bill-becomes}, the
analytic problem thus becomes identifying which amendments and
portions of the initial bill become the final bill. We treat this as a
document retrieval problem, wherein we wish to query for a document
$D_{2,i}$ and retrieve its most likely match from $D_2,
A_J$. \footnote{This can, alternatively, be treated as a
subset of the plagiarism problem, wherein we know with probability
approaching one that $D_2$ has plagiarized content from $D_1$ and
$A_J$. As with other versions of plagiarism detection, the problem
then becomes establishing an effective and efficient comparison
strategy.}

We implement a generic comparison strategy in four steps:
\begin{enumerate}
\item Transform each section of $D_2$, $D_1$, and $A_J$ into a
  bag-of-words vector-space representation in which features are
  n-grams of potentially heterogeneous length
\item Construct the pairwise similarity, using some similarity metric,
  between each section of $D_2$ and all sections in $D_1$ and $A_J$.
\item Use nearest-neighbor matching with replacement to establish
  the closest match between each section of $D_2$ and a section in
  $D_1$ and a candidate amendment in $A_J$
\item Pick the closest match of the pair of potential matches $d_i$
  and $a_j$
\end{enumerate}

Implementation of step (1) occurs via the \texttt{tm} package for R
\citep{meyer2008text}, and uses the Weka tokenizer for term
tokenization \citep{hall2009weka}. This infrastructure provides users
the ability to specify whether words should be stemmed; whether
punctuation, excess whitespace or stopwords should be removed; the
minimum and maximum n-gram length used for constructing the vector
space representation of text, and whether the term list should be
filtered on the basis of one of several metrics including
term-frequency and term frequency / inverse document frequency
measures.

The \texttt{leghist} package comes with several alternative distance
metrics for use in step (2). For token-based distance metrics, we
implement a standard cosine similarity, a length-weighted cosine
similarity, and a length-weighted joint information distance measure
as described by \cite{hoad2003methods}. Each is implemented as matrix
operation, permitting rapid pairwise distance calculations even for
relatively large ($\sim$1000+ amendment) corpora. In plagiarism
applications, length-sensitive distance metrics have performed well in
identifying correct documents from otherwise quite similar candidate
matches. However, the matching process here appears to favor the fuzzy
matches provided by the length-insensitive cosine similarity. This
may, as we will discuss, be due to the somewhat arbitrary choice of
what constitutes a document section, and its instability over
time. Finally, the function interfaces allow the user to supply any
distance function so long as it returns a correctly-formatted distance
matrix. We also implement an interface to string-based metrics like
the Levenshtein edit distance \citep{elmagarmid2007}, though these
metrics are far more computationally-intensive than the cosine
similarity or set-based measures like the Jaccard index.

Step (3) uses simple nearest neighbor matching with replacement to
identify the most likely candidate matches from both the initial
document and the set of proposed amendments. Both matches are returned
in an intermediate step to permit users to identify the connection to
the initial bill and to the potential amendments. Finally, under the
assumption that the similarity metric is consistent and well-ordered,
we can then in step (4) compare this pair of potential matches to
determine the best potential match from all candidate matches.


\subsection{Learning the distance threshold}
\label{sec:learn-dist-thresh}

Given the distance matrix, construction of the composite matched bill
requires a threshold value for accepting or rejecting
matches. Selection of an optimum threshold value will depend on the
specifics of the legislative process behind each bill. For instance,
the right threshold value for a bill that was modified only via the
amendment process may be much lower than a threshold value valid for a
bill that went through a negotiation or reconciliation process. While
our experiments suggest that values on the order of 0.3 perform
reasonably well, we recommend that users base threshold selection on
the performance of the matching algorithm in their particular
case. For instance, legislative procedures known to have significant
negotiated content may benefit from a higher threshold value, while
those with very little negotiated content might require a threshold
value close to 0.

Identification of the context-specific threshold value constitutes a
supervised learning problems. To facilitate identification of
context-specific threshold values, we provide tools that allow the
user to hand-code a subset of a given bill and derive the optimum
threshold value based on the algorithm's performance against that
hand-coded data. In practice, coding of 30\% of a bill appears
sufficient to derive acceptable threshold values.

Figure \ref{fig:bill-accuracy-curves} provides an example of the
performance of the document retrieval process as established by this
supervised approach. For each of four different bills, 30\% of
the final bill was hand-matched to the most semantically appropriate
sources in either the original draft of the bill or
subsequently-proposed amendments. Based on this sub-sample, we
established the optimum threshold based on both overall accuracy and
the precision-recall trade off between Type 1 (precision) and Type 2
(recall) errors. In all cases, the optimum threshold lay between
0.3-0.45. The variation in the optimum threshold setting appears to
derive from two primary sources: the degree of negotiation (as opposed
to formal amendment) in the legislative process itself; and the amount
of structural change that occurred when amendments were incorporated
into formal legislative language.

\begin{figure}[ht]
  \centering
  \includegraphics[width=0.45\textwidth]{ets_2003_accuracy_curves}
  \includegraphics[width=0.45\textwidth]{ets_2007_accuracy_curves}
  \includegraphics[width=0.45\textwidth]{intl_2003_accuracy_curves}
  \includegraphics[width=0.45\textwidth]{rese_2007_accuracy_curves}
  \caption{Accuracy assessment curves for 30\% training samples in
    each of four different bills. Bills were drawn from the European
    Union in years 2001, 2003, and 2007. Optimum threshold values
    ranged from 0.3-0.4 in all cases.}
  \label{fig:bill-accuracy-curves}
\end{figure}

\subsection{Bill synthesis}
\label{sec:bill-synthesis}

Based on the pairwise matches between sections in the final bill and
the set of potential sources for the final version of those sections,
we can construct a synthetic bill for comparison purposes. The package
provides tools to construct a side-by-side comparison of the final
bill sections to their candidate matches. If the user supplies a
non-zero threshold value as described in section
\ref{sec:extens-negot-cont}, that
value is used to determine whether the actual match
should be used. If the threshold value is not met, then the final
document section is matched to itself. 


\subsection{Semantic clustering and impact analysis}
\label{sec:semant-clust-impact}

Finally, the legislative process rarely amends bills at
random. Rather, specific sections or issue areas within bills will
often prove more contentious than others. The user may wish to
identify these areas on the basis of accepted and rejected amendments,
sections of the original bill that persist to the final bill, and
sections of the final bill for which no match was identified. 

We provide a ready interface between the matched bill and the
\texttt{topicmodels} package for R that enables topic modeling of
these different slices of the matched bill. As described by
\cite{blei2003latent}, topic modeling assumes a latent
structure within a set of documents. Each document--in this case, a
bill section--is assumed to encompass one or more of an unobserved set
of latent topics that underpin the entire document collection. Each
topic, in turn, is represented by distribution over a
set of terms. Latent Dirichlet Allocation provides a Bayesian method
for inferring topics from empirical word distributions, and assigning
documents to those topics on the basis of document-specific
distributions. 

\subsection{Visualization}
\label{sec:visualization}

Finally, \texttt{leghist} provides, via the \texttt{WriteSideBySide}
functionality, for side-by-side comparison of the actual final
bill and its synthetic match in a two-column \LaTeX\ document, which
can be directly compiled into a PDF. Within that document, each
paragraph of the final bill is shown side-by-side with its
match. Words differences between the paragraphs are highlighted as
colored text. Margin notes indicate the origin of the matched segment
and its similarity or distance measure compared with the section to
which it is matched.


\section{Vignette: the 2009 European Renewable Energy law }
\label{sec:vign-2009-europ}

To illustrate each of these tools and their technical underpinnings,
we now provide a vignette based on actual legislation as passed by the
European Parliament. In 2008, the European Commission proposed a new
round of reforms to renewable energy policy. This reform followed
on the 2001 Renewable Energy Directive. This series of legislation set
Europe-wide and country targets for renewable energy adoption and
established a European framework for renewable energy investment and
subsidy. The reforms also played an important role in the multi-part
bargain on energy and climate change introduced in the Third Climate
and Energy Package.

The original bill in this case was a package of amendments to the 2001
legislation proposed by the European Commission as Commission document
COM 2008 0019. Based on this document, the European Parliament
submitted a range of amendments in the report tabled for the plenary
session in Strasbourg. Subsequently, the Parliament and the Council
agreed on a final bill, which passed both bodies. The Commission did
not contest any of the agreed-upon changes. The final bill was
adopted as Directive 2009 (28).\footnote{A full procedural record can
  be found at
  \url{http://www.europarl.europa.eu/oeil/popups/ficheprocedure.do?reference=2008/0016(COD)&l=en}. Referenced
10 June 2012.} 

This section demonstrates the operation of the \texttt{leghist}
package in analyzing the amendment process for this legislation. We
treat the initial bill as the 2001 legislation which the Commission
proposed to change. Amendments then constitute the Commission proposal
and the amendments contained in the First Reading
Report. 

We begin by reading in the bills and amendments as plain text. In each
case, the bills had been parsed from original documents available from
the European Legislative Observatory. 

\begin{lstlisting}[language=R, numbers=none]
ep.first.2007 <- read.csv("./2007/txt/ep_first_reading_report.csv",
                          header=TRUE,
                          stringsAsFactors=FALSE
                          )

bill.initial.2007 <- readLines("./2001/orig/directive_2001_77_ec.txt")
commission.proposal.2007 <- readLines("./2007/txt/com_2008_0019.txt")
bill.final.2007 <- readLines("./2007/txt/directive_2009_28_ec.txt")

committees.2007 <- c(rep("Parliament 1st reading",
                         nrow(ep.first.2007)),
                     rep("Commission",
                         length(commission.proposal.2007))
                     )

\end{lstlisting}


\subsection{Distance estimation and matching}
\label{sec:dist-estim-match}


The documents are then transformed into a single document-term
matrix containing a bag-of-words representation of each
section of the bills and proposed amendments. Each row of the matrix
represents one section of one document. Columns represent unique terms
in those documents as term frequencies for those terms in each
document section. In this case, we do not stem the terms, but do remove
English stopwords, excess whitespace, and punctuation. We also use
a range of n-grams, to preserve the semantic resolution of
consecutive-word combinations.

\begin{lstlisting}[language=R, numbers=none]
doc.list.2007 <- CreateAllVectorSpaces(bill.initial.2007,
                                       bill.final.2007,
                                       c(ep.first.2007$text,
                                         commission.proposal.2007
                                         ),
                                       ngram.min=1,
                                       ngram.max=3,
                                       stem=FALSE,
                                       rm.stopwords=TRUE,
                                       rm.whitespace=TRUE,
                                       rm.punctuation=TRUE,
                                       filter=NULL,
                                       filter.thres=NULL,
                                       weighting=weightTf
                                       )

\end{lstlisting}

The \texttt{doc.list.2007} is a R \texttt{list} containing four components::
\begin{itemize}
\item \texttt{vs.out}, the document-term matrix represented as a
  sparse \texttt{dgCMatrix} object \citep{Bates2012}
\item \texttt{idx.final}, an integer vector of row indices
  representing documents in the final bill
\item \texttt{idx.initial}, an integer vector of row indices
  representing documents from the initial bill
\item \texttt{idx.amendments}, an integer vector of row indices
  representing documents from the proposed amendments
\end{itemize} 

Using the document-term matrix, the \texttt{MapBills} function
generates a distance or similarity matrix between all sections of the
final bill and their candidate origins in the initial bill and
proposed amendments. Here, we use
\texttt{CosineMat}, a version of the cosine similarity
optimized for this use case by using matrix algebra to compute all
distances simultaneously. However, \texttt{MapBills} will accept any function that returns a
distance matrix of the $N_{final} \times N_{compare}$.

\begin{lstlisting}[language=R, numbers=none]
map.bills.2007.cos <- MapBills(doc.list.2007,
                               distance.fun="CosineMat"
                               )
\end{lstlisting}

\texttt{MapBills} returns an R data frame with five columns: the
section index of the final bill, and the section index and distance
measure of both the original bill match and the amendment match. We
can see a sample of that data frame here:

\begin{verbatim}
  bill2.idx bill1.idx bill1.dist amend.idx amend.dist
1         1         1        NaN         1        NaN
2         2         6  0.2377064       981  0.7376436
3         3         3  0.7177070      1082  0.3880453
4         4         4  0.5805601      1083  0.6782620
5         5        24  0.1116871        94  0.8236878
\end{verbatim}


\subsection{Learning the similarity threshold}
\label{sec:learn-simil-thresh}

To select the optimum threshold value as discussed in section
\ref{sec:learn-dist-thresh}, \texttt{leghist} 
provides a 2-step workflow. First, the \texttt{RunEncoder} routine allows the user to
encode a randomly-sampled portion of the final bill. It presents the
user with a set of candidate matches, based on likely matches from
both the initial bill and amendments. The user provides the index of
the best match, or \texttt{NA} if no valid match is
available.\footnote{We note that users who do not want to use the
  automated matching tools may still use the encoding tools to
  hand-match amendments. The tools reduce the set of possible matches
  to a set of likely matches based on the same vector space language
  representation used by the automated match algorithms, thereby
  reducing the search space for hand-coded matches.} Figure
\ref{fig:encoder-interface} provides a sample of the 
\texttt{RunEncoder} interface.

Second, based on the output of \texttt{RunEncoder},
\texttt{LearnThreshold} computes accuracy measures for a synthetic
bill based on a vector of threshold values. Based on those accuracy
measures, \texttt{LearnThreshold} returns optimum threshold values
from the sequence. We recommend that the
threshold sequence be specified at a fairly granular level, such as
\texttt{seq(0, 0.5, 0.005)}. The optimum threshold value may be identified in
two ways based on the \texttt{type} argument to
\texttt{LearnThreshold}. With \texttt{type=overall}, the aggregate
accuracy of the matched pairs is estimated, and the threshold value
corresponding to the highest accuracy returned. With
\texttt{type=tradeoff}, the threshold value at the intersection of the
the precision/recall tradeoff curves. 

Figure \ref{fig:learn-threshold-output} shows the
output of \texttt{LearnThreshold} for both \texttt{overall} and
\texttt{tradeoff} settings.We can see that the optimum choice of
threshold value is about 0.45.

\begin{figure}[ht]
  \centering
  \includegraphics[width=0.75\textwidth]{rese_2007_accuracy_curves.pdf}
  \caption{Accuracy curves as returned from the \texttt{LearnThreshold} function. Notice that the optimum threshold value is closer to the accuracy curve than the type 1/2 tradeoff curve intersection, owing to the relative flatness of the tradeoff curves.}
  \label{fig:learn-threshold-output}
\end{figure}

\subsection{Generating and visualizing the synthetic bill}
\label{sec:gener-synth-bill}

\texttt{GetLikelyComposite} provides functionality for building up the
match to the final bill from the output of \texttt{MapBills} and the
learned distance threshold. It takes a series of arguments: the output
from MapBills, the plain text used to create the original
\texttt{doc.list} object, and labels that indicate who originated the
amendments (in this case, the Parliament at First Reading, and the
European Commission). For any section of the final bill,
\texttt{GetLikelyComposite} will return the best match if the
similarity threshold exceeds \texttt{dist.threshold}; otherwise, it
will insert the same section of the "final" bill as the best
match. This can be used to identify sections of the bill with
uncertain origins, which perhaps arose outside of the formal amendment
process.


\begin{lstlisting}
rese.composite.2007.cos <- GetLikelyComposite(map.bills.2007.cos,
                                              bill.initial.2007,
                                              bill.final.2007,
                                              c(ep.first.2007$text,
                                                commission.proposal.2007
                                                ),
                                              committees.2007,
                                              filter="max",
                                              dist.threshold=0.45
                                              )
                       
\end{lstlisting}


\subsection{Impact analysis}
\label{sec:semant-impact-analys}

To identify the semantic and substantive influence of the EU
legislative actors, we make use of the topic modeling capabilities in the
\texttt{topicmodels} package \citep{grun2011topicmodels}. These, in
turn, provide R interfaces to the Latent Dirichlet Allocation methods
implemented in \cite{blei2003latent} and
\cite{blei2006correlated}. These methods infer a set of topics from
the term distributions over a set of documents, on the assumption that
topics are differentiated by different distributions over a set of
terms.  The \texttt{leghist} package
provides a structured interface to cluster legislative text using
these tools and tabulate the output.

We first identify new substantive areas introduced into the final bill
during the legislative process. The \texttt{ModelTopics} function
provides a means for modeling the topic distribution for the entire
corpus. We recommend that users select the optimum $k$ value on the basis of
both formal model selection criteria and substantive topic
coherence. \cite{wallach2009evaluation} provide a range of formal
methods; \texttt{topicmodels} also implements log-likelihood and
perplexity measures. 

\begin{lstlisting}[language=R, numbers=none]
## Subset the columns to include only 1-grams
addl.stopwords <- c("december", "paragraph", "article", "january",
                    "march", "june", "commission", "parliament", "community"
                    )
idx.to.keep <- leghist:::GetNgramIdx(colnames(doc.list.2007$vs.out),
                                     1
                                     )
rese.1gram.2007 <- doc.list.2007$vs.out[,idx.to.keep]

## Here k was chosen to maximize the log-likelihood of 
## the model on a held-out 10% sample of the full 
## document-term matrix
rese.full.model <- ModelTopics(rese.1gram.2007,
                         idx=c(doc.list.2007$idx.final,
                           doc.list.2007$idx.initial,
                           doc.list.2007$idx.amendments),
                         k=26,
                         sampling.method="VEM",
                         topic.method="LDA",
                         control=list(
                           var=list(tol=10^-4),
                           em=list(tol=10^-3)
                           ),
                         n.terms=10,
                         addl.stopwords=addl.stopwords,
                         na.rm=FALSE
                         )  
\end{lstlisting}



With the full model, we can now match topics to individual document
sections, their status as accepted or rejected in the final bill, and
the legislative actor that proposed them.

\begin{lstlisting}
rese.topics.2007 <- CollateTopicDtm(rese.composite.2007.cos,
                                    rese.full.model,
                                    doc.list.2007,
                                    committees.2007
                                    )

impact <- EstimateSourceImpact(rese.topics.2007)
\end{lstlisting}

\texttt{CollateTopicDtm} returns two matrices with topic and status
information. The first, \texttt{doc.topic} maps the entire document-term matrix to the
committee that originated each row, the status of that document
section, and the topic assigned by the model. The second,
\texttt{composite.topic}, is a matrix of the same form as
\texttt{GetLikelyComposite}, augmented with the topic assigned to both
the final bill section and its assigned match in the original bill or
amendments. Sample output is shown below.

\begin{verbatim}
> rese.topics.2007$doc.topic[1:5,]
            source committee idx status topic
result.1 doc.final     final   1    acc  <NA>
result.2 doc.final     final   2 redund    22
result.3 doc.final     final   3 redund     6
result.4 doc.final     final   4 redund     3
result.5 doc.final     final   5 redund    16
\end{verbatim}

\texttt{EstimateSourceImpact} then operates on these results to return
tables identifying semantic additions to the original proposed
legislation. It identifies which topics in the final bill are not
present in the initial, and tabulates documents assigned to those
topics by legislative actor. It also breaks down the quality of the
match, estimated by the similarity metric, as a measure of how well
For the 2007 renewable energy bill, we can easily see that the
Parliament had substantially more influence over topics 1, 7, and
11. Looking at the terms returned from \texttt{ModelTopics}, it's
clear that the Parliament's influence in new policy domains came
primarily in areas surrounding the impact of biofuels on third
countries and on land use; and secondarily on  influence cam largely in amendments
governing national renewable energy action plans, the impact of
biofuels on third countries and agricultural land, and regulatory
compliance procedures. 

\begin{verbatim}
> round(impact$prop.origin.topic, 2)
## Proportion of bill in each topic by contributing 
## legislative actor
                        Topic
Source                     1   11   12   14  16   18   19    2   23   26    5
  Commission             0.0 0.33 0.31 0.58 0.5 0.46 0.75 0.57 0.20 0.75 0.43
  Final                  0.5 0.27 0.56 0.27 0.2 0.15 0.00 0.14 0.67 0.08 0.36
  Parliament 1st reading 0.5 0.40 0.12 0.15 0.3 0.38 0.25 0.29 0.13 0.17 0.21

                        Topic
Source                      7 9
  Commission             0.20 1
  Final                  0.47 0
  Parliament 1st reading 0.33 0

      Topic 1      Topic 11      Topic 18         Topic 2         Topic 7          
 [1,] "countries"  "land"        "training"       "materials"     "accordance"     
 [2,] "third"      "raw"         "action"         "raw"           "regulatory"     
 [3,] "country"    "purposes"    "national"       "emissions"     "adopted"        
 [4,] "convention" "material"    "local"          "cultivation"   "measures"       
 [5,] "impact"     "biofuels"    "plans"          "values"        "referred"       
 [6,] "concerning" "obtained"    "regional"       "agricultural"  "procedure"      
 [7,] "community"  "provided"    "administrative" "production"    "elements"       
 [8,] "labour"     "forest"      "including"      "gas"           "essential"      
 [9,] "biofuel"    "significant" "planning"       "greenhouse"    "infrastructure" 
[10,] "treaties"   "bioliquids"  "bodies"         "list"          "scrutiny"       
\end{verbatim}

We also provide a hierarchical interface, which first clusters document
sections into semantically-related groups, and then clusters the
documents inside each cluster into semantic sub-groups. This
functionality implements the following intuition: consider a group of
amendments and bill sections classified as "procedural" and identified
because of the predominance of terms like "audit", "budget",
"monitor", and "oversight". We would expect that, within that cluster,
we would observe that different amendments focus on different aspects
of these procedural elements of legislation. Hence sub-clustering
provides a means of breaking down bills and amendments into a sensible
2-level semantic hierarchy. Analysts could, of course, iterate this
process for any desired level of hierarchy. 

We also provide a means of tabulating the committee contributions by
both primary and secondary topic areas. The
\texttt{ctab.amend.hierarchy} function takes the output from
\texttt{model.amend.hierarchy} and returns the nested cross-tabulation
of committee contributions to each semantic cluster and sub-cluster. 

\begin{lstlisting}[language=R, numbers=none]
## Define the number of primary topics
k.main <- 8
rese.topics.2007 <- 
   ModelAmendHierarchy(doc.list.2007,
                       rese.composite.2007,
                       k=c(k.main, rep(5, k.main)),
                       addl.stopwords=c("article",
                         "paragraph", "follow", "replace",
                         "insert",
                         "ensure", "regulation",
                         as.character(0:9),
                         "commission",
                         "directive", "annex",
                         "parliament",
                         "council",
                         "alia", "whereas",
                         "january", "december", "july",
                         "deleted", "added", "amend",
                         "draft", "http", "mso",
                         "allowincell", "textbox",
                         "endiftextbox", "div"),
                       n.terms=10,
                       ngram=1,
                       sampling.method="VEM",
                       topic.method="CTM",
                       control=list(
                         var=list(tol=10^-4),
                         em=list(tol=10^-3),
                         seed=2342),
                       sparseness.probs=c(0.01, 0.999)
                       )


rese.topic.proportions.2007 <-
  CtabAmendHierarchy(rese.topics.2007,
                     rese.composite.2007,
                     committees.2007,
                     doc.list.2007,
                     tab.idx=2
                     )

\end{lstlisting}

Example output is shown below. We can quickly observe that the
Commission had, for instance, far more influence on amendments related
to topic 5, on specification and definition of renewable liquid fuels,
than the Parliament. Likewise, the Parliament exercised greater
influence over amendments in topic 6, governing the transfer of
renewable energy production credits among EU countries and between EU
countries and third parties. Likewise, we can see that within the
general set of regulations governing grid operators (topic 7),
different groups of amendments governed connection requirements,
administrative licensing and planning, emissions reduction and
renewable energy accommodation, and energy efficiency. These
inferences proceed in straightforward fashion from the
\texttt{leghist} workflow combining
an automated approach to legislative history tracking, and readily
available tools for semantic clustering of text. 

\begin{verbatim}
## Top ten words by influence for the 8-topic model
      [,1]             [,2]            [,3]            [,4]            
 [1,] "bioliquids"     "training"      "consumption"   "raw"           
 [2,] "countries"      "solar"         "share"         "support"       
 [3,] "significant"    "accordance"    "heat"          "materials"     
 [4,] "third"          "benefits"      "final"         "sustainability"
 [5,] "provided"       "adopted"       "target"        "schemes"       
 [6,] "sustainability" "share"         "overall"       "development"   
 [7,] "environmental"  "procedure"     "means"         "bioliquids"    
 [8,] "food"           "regulatory"    "mandatory"     "material"      
 [9,] "protection"     "certification" "set"           "environmental" 
[10,] "sustainable"    "appropriate"   "installations" "report"        
      [,5]        [,6]           [,7]           [,8]          
 [1,] "oil"       "origin"       "operators"    "fuels"       
 [2,] "values"    "transfer"     "table"        "emission"    
 [3,] "ethanol"   "guarantees"   "grid"         "resources"   
 [4,] "waste"     "competent"    "carbon"       "specific"    
 [5,] "process"   "accounting"   "require"      "savings"     
 [6,] "wood"      "body"         "distribution" "based"       
 [7,] "default"   "guarantee"    "procedures"   "efficiency"  
 [8,] "vegetable" "certificates" "transmission" "cogeneration"
 [9,] "biogas"    "issued"       "producers"    "information" 
[10,] "biofuel"   "compliance"   "rules"        "changes"
\end{verbatim}

\newpage
\begin{verbatim}
## Distribution of amendment acceptance and rejection
## by legislative actor and topic number
                            acc   rej
                                     
1 Commission               50.0   5.5
  Parliament 1st reading   50.0  94.5
2 Commission               61.8   8.0
  Parliament 1st reading   38.2  92.0
3 Commission               56.5   6.2
  Parliament 1st reading   43.5  93.8
4 Commission               54.5   4.5
  Parliament 1st reading   45.5  95.5
5 Commission               77.8  18.4
  Parliament 1st reading   22.2  81.6
6 Commission               36.1  21.2
  Parliament 1st reading   63.9  78.8
7 Commission               54.5   6.3
  Parliament 1st reading   45.5  93.7
8 Commission              100.0   6.7
  Parliament 1st reading    0.0  93.3
\end{verbatim}

\begin{verbatim}
## Primary and secondary topic features for topic 7 
[[7]]
[[7]]$terms.primary
 [1] "operators"    "table"        "grid"         "carbon"       "require"     
 [6] "distribution" "procedures"   "transmission" "producers"    "rules"       

[[7]]$terms.secondary%$
      Topic 1        Topic 2     Topic 3      Topic 4       Topic 5         
 [1,] "operators"    "table"     "procedures" "carbon"      "administrative"
 [2,] "grid"         "filled"    "equipment"  "wind"        "certification" 
 [3,] "costs"        "stroked"   "systems"    "economic"    "authorisation" 
 [4,] "transmission" "rectrect"  "conversion" "operators"   "licensing"     
 [5,] "distribution" "carbon"    "promote"    "information" "local"         
 [6,] "producers"    "unit"      "achieve"    "roads"       "planning"      
 [7,] "require"      "measured"  "efficiency" "offshore"    "applications"  
 [8,] "connection"   "stock"     "stock"      "development" "regional"      
 [9,] "regions"      "mass"      "carbon"     "impact"      "body"          
[10,] "rules"        "molecular" "annualised" "priority"    "established" 
\end{verbatim}

\subsection{Visualizing output}
\label{sec:visu-proc-flow}

\texttt{leghist} provides a visualization framework for assessing
the quality of the composite bill and the flow of text from
legislative actors to final status. Based on the
\texttt{GetLikelyComposite} output, \texttt{WriteSideBySide} provides
an interface for building a 2-column \LaTeX\ document showing the final
bill and its matched origins. Margin notes indicate the source of the
matched paragraph, its index number, and the distance or similarity
measure for its match to the target paragraph in the final
bill. Sample PDF output is shown in figure \ref{fig:pdf-output}.

\begin{lstlisting}[language=R]
rese.sbs.2007 <- WriteSideBySide(rese.composite.2007.cos,
                                 bill.final.2007,
                                 cavs.out=doc.list.2007,
                                 file.out="ep_2007_rese_cos.tex",
                                 dir.out="./2007/tex/",
                                 pdflatex=TRUE
                                 )
\end{lstlisting}

\begin{figure}[ht]
  \centering
  \includegraphics[width=0.75\textwidth]{sample_pdf_output}
  \label{fig:pdf-output}
  \caption{Sample PDF output from the \texttt{WriteSideBySide} function, for the 2009 European energy market reform legislation.}
\end{figure}

 


\section{Accuracy trials}
\label{sec:accuracy-trials}

The analyses that \texttt{leghist} makes available are only as good as the
quality of the matches. We assess the accuracy of the matching
algorithm against a human-matched version of the same bill. Using the 
\texttt{RunEncoder} interface provided in \texttt{leghist} and shown
in figure \ref{fig:encoder-interface}, we had-coded matches for all
380 sections in the final bill. We then compared the algorithmic match
to the manual match using the \texttt{LearnThreshold}
tools. Figure \ref{fig:rese-accuracy-test} shows the results of those
comparisons. For a threshold value of 0.45, the algorithm correctly
identified and matched amendments in 84\% of cases. The rate of
correct source + index matching was lower. The discrepancy between
identifying something as an amendment, versus \textit{which}
amendment, is largely due to the presence of ties. If
more than one committee submits an amendment with identical language,
the algorithm breaks ties based on order in the amendment data. Manual
matching may return the same substantive match, but a different
index.


\begin{figure}[ht]
  \centering
  \includegraphics[width=\textwidth]{encoder}
  \caption{Sample of the \texttt{RunEncoder} interface, running in the R console. }
  \label{fig:encoder-interface}
\end{figure}

\begin{figure}[ht]
  \centering
  \includegraphics[width=0.75\textwidth]{rese_2007_accuracy_curves_full}
  \caption{Algorithm accuracy as a function of the similarity threshold value. Baseline for comparison is a fully human-encoded bill. Source accuracy indicates where the algorithm identified the same source (amendment, original bill, or final bill) as the human coder. Source + index accuracy indicates where the algorithm found both the same source and the same index value (e.g. Amendment 24) as identified by the human coder.  }
  \label{fig:rese-accuracy-test}
\end{figure}


\section{Conclusions}
\label{sec:conclusions}

The \texttt{leghist} package provides a range of utilities for
identifying the origin of sections of a final piece of legislation
given a set of possible sources. It further provides tools to analyze,
cluster, and visualize these matches. We hope that these tools will
prove useful to analysts of legislation and other documents in which
the process of document preparation is as important as the result. 

We anticipate several useful extensions. A more elaborate document
retrieval process would perhaps improve on the quality of the
matches. Likewise, we might imagine a further set of algorithms that
help determine the optimal level at which to compare the final and
source documents, on the basis of some loss function. Furthermore, we
may be able to exploit the distribution of the distance metric for
each match to assess match quality, inferring the right threshold level
$T$ from data rather than from manual user input; that would improve
\texttt{leghist}'s utility when hand-coding was unfeasible, as with
very large document corpora. Finally, the
process presented here makes no use of positional information; the
fact that amendments are proposed to specific sections of a document
gets lost in the parsing stage. While the process of drafting
legislation destroys the value of a specific index, it does not
necessarily mean that a paragraph once at the center of a bill now
appears at the very end. Exploiting that information to restrict the
set of potential matches may improve overall match quality. Finally,
we are interested in, and hope to implement in future versions,
measures of semantic, rather than textual, similarity to quantify
differences in the degree of change implied by competing amendments to
the same section of a bill. 



\bibliography{/Users/markhuberty/Documents/Research/Papers/leghist/bib/leghist}
\bibliographystyle{apalike}
\end{document}
