% Created 2011-11-02 Wed 08:53
\documentclass[11pt]{article}
\usepackage[utf8]{inputenc}
\usepackage[T1]{fontenc}
\usepackage{fixltx2e}
\usepackage{graphicx}
\usepackage{longtable}
\usepackage{float}
\usepackage{wrapfig}
\usepackage{soul}
\usepackage{t1enc}
\usepackage{textcomp}
\usepackage{marvosym}
\usepackage{wasysym}
\usepackage{latexsym}
\usepackage{amssymb}
\usepackage{hyperref}
\tolerance=1000
\providecommand{\alert}[1]{\textbf{#1}}

\title{design}
\author{Mark Huberty}
\date{02 November 2011}

\begin{document}

\maketitle

\setcounter{tocdepth}{3}
\tableofcontents
\vspace*{1cm}
\section{Design notes}
\label{sec-1}

\begin{enumerate}
\item Conceptual framework:
   For a piece of legislation L that goes through V versions L$_v$, and is
   subject to A proposed amendments, we would like to know how that
   legislation changes across the legislative cycle, which amendments
   were adopted, and whether they represent changes to substantive or
   administrative aspects of the law (i.e., what the law is intended
   to accomplish, versus how it should accomplish it).
\item Automating this process would ideally return, for any two versions
   L$_1$ and L$_2$, several pieces of information:

\begin{itemize}
\item A mapping of the sections of L$_1$ to their new location in L$_2$
\item A probabilistic indicator of the confidence of the map
\item A summary of the new content added
\item A summary of the content taken away
\item An indication of whether the changes were substantive or
     administrative
\end{itemize}

\item 
\end{enumerate}

   

\end{document}