% Created 2011-11-04 Fri 09:46
\documentclass[11pt]{article}
\usepackage[utf8]{inputenc}
\usepackage[T1]{fontenc}
\usepackage{fixltx2e}
\usepackage{graphicx}
\usepackage{longtable}
\usepackage{float}
\usepackage{wrapfig}
\usepackage{soul}
\usepackage{t1enc}
\usepackage{textcomp}
\usepackage{marvosym}
\usepackage{wasysym}
\usepackage{latexsym}
\usepackage{amssymb}
\usepackage{hyperref}
\tolerance=1000
\providecommand{\alert}[1]{\textbf{#1}}

\title{Quant/Qual research possibilities}
\author{Mark Huberty}
\date{04 November 2011}

\begin{document}

\maketitle

\setcounter{tocdepth}{3}
\tableofcontents
\vspace*{1cm}

\section{Project possibilities}
\label{sec-1}
\subsection{Automated mapping of legislative history}
\label{sec-1_1}

   Project goals: 
\begin{enumerate}
\item Provide a method for algorithmic mapping of final
      legislation to the initial bill + amendments, based on text
      similarity measures.
\item Provide easy access to methods to model the content of what is
      taken out or added to legislation during the political process
\item (Potentially) provide some ``importance'' measure for various
      pieces of the text
\item (Potentially) provide a categorization mechanism for breaking
      out legislation between its ``substantive'' and ``administrative''
      (that is, what the legislation does, and how it does it) components
\end{enumerate}


   This project would require some experimentation with text modeling
   in R as an input to the design process. We would probably use EU
   legislation that is easily available online in HTML format.
      
\subsection{Analysis of US public opinion on global warming based on local economic factors}
\label{sec-1_2}

   Background: I'd looked at this a few years ago, to little
   effect. One hypothesis on public opinion and global warming is that
   it's a function of a locality's dependence on energy or fossil
   fuels. The crude version of this states that West Virginians (a big
   coal state) won't vote for emissions reduction, because that would
   (obviously) cut the market for coal. 

   There's some work (see Matt Kahn @ UCLA) on linking emissions
   themselves to public opinion. But that's confounded by various
   things, like opportunities for efficiency improvements. I'd be
   interested to see whether we could get enough data on local
   patterns of energy use and industrial output that we could look at
   the direct correlation of economic factors with local attitudes on
   climate change.

   See Craig \& Kahn, NBER Working Paper no 14963, and here:
   \href{http://goo.gl/gwPMr}{http://goo.gl/gwPMr}

   Alternative, we may wish to try this out in the EU instead, but the
   data issues there may be more substantial. 
\subsection{Analysis of the role of sectoral positioning on climate policy action}
\label{sec-1_3}

   Background: Georg Zachmann and I have used Hidalgo's concept of the
   product space to do some econometric analysis of the origins and
   evolution of comparative advantage in green energy goods. There's a
   related question, as to whether people see their interests
   connected with actual or potential green energy industries and are
   OK with climate change policy as a result. 

   I've not thought this through at all, so there are no guarantees
   that this has any legs to it.

\end{document}